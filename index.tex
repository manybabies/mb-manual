% Options for packages loaded elsewhere
% Options for packages loaded elsewhere
\PassOptionsToPackage{unicode}{hyperref}
\PassOptionsToPackage{hyphens}{url}
\PassOptionsToPackage{dvipsnames,svgnames,x11names}{xcolor}
%
\documentclass[
  letterpaper,
  numbers=noenddot,
  DIV=11,
  oneside]{scrreprt}
\usepackage{xcolor}
\usepackage[left=1in,marginparwidth=2.0666666666667in,textwidth=4.1333333333333in,marginparsep=0.3in]{geometry}
\usepackage{amsmath,amssymb}
\setcounter{secnumdepth}{5}
\usepackage{iftex}
\ifPDFTeX
  \usepackage[T1]{fontenc}
  \usepackage[utf8]{inputenc}
  \usepackage{textcomp} % provide euro and other symbols
\else % if luatex or xetex
  \usepackage{unicode-math} % this also loads fontspec
  \defaultfontfeatures{Scale=MatchLowercase}
  \defaultfontfeatures[\rmfamily]{Ligatures=TeX,Scale=1}
\fi
\usepackage{lmodern}
\ifPDFTeX\else
  % xetex/luatex font selection
\fi
% Use upquote if available, for straight quotes in verbatim environments
\IfFileExists{upquote.sty}{\usepackage{upquote}}{}
\IfFileExists{microtype.sty}{% use microtype if available
  \usepackage[]{microtype}
  \UseMicrotypeSet[protrusion]{basicmath} % disable protrusion for tt fonts
}{}
\makeatletter
\@ifundefined{KOMAClassName}{% if non-KOMA class
  \IfFileExists{parskip.sty}{%
    \usepackage{parskip}
  }{% else
    \setlength{\parindent}{0pt}
    \setlength{\parskip}{6pt plus 2pt minus 1pt}}
}{% if KOMA class
  \KOMAoptions{parskip=half}}
\makeatother
% Make \paragraph and \subparagraph free-standing
\makeatletter
\ifx\paragraph\undefined\else
  \let\oldparagraph\paragraph
  \renewcommand{\paragraph}{
    \@ifstar
      \xxxParagraphStar
      \xxxParagraphNoStar
  }
  \newcommand{\xxxParagraphStar}[1]{\oldparagraph*{#1}\mbox{}}
  \newcommand{\xxxParagraphNoStar}[1]{\oldparagraph{#1}\mbox{}}
\fi
\ifx\subparagraph\undefined\else
  \let\oldsubparagraph\subparagraph
  \renewcommand{\subparagraph}{
    \@ifstar
      \xxxSubParagraphStar
      \xxxSubParagraphNoStar
  }
  \newcommand{\xxxSubParagraphStar}[1]{\oldsubparagraph*{#1}\mbox{}}
  \newcommand{\xxxSubParagraphNoStar}[1]{\oldsubparagraph{#1}\mbox{}}
\fi
\makeatother


\usepackage{longtable,booktabs,array}
\usepackage{calc} % for calculating minipage widths
% Correct order of tables after \paragraph or \subparagraph
\usepackage{etoolbox}
\makeatletter
\patchcmd\longtable{\par}{\if@noskipsec\mbox{}\fi\par}{}{}
\makeatother
% Allow footnotes in longtable head/foot
\IfFileExists{footnotehyper.sty}{\usepackage{footnotehyper}}{\usepackage{footnote}}
\makesavenoteenv{longtable}
\usepackage{graphicx}
\makeatletter
\newsavebox\pandoc@box
\newcommand*\pandocbounded[1]{% scales image to fit in text height/width
  \sbox\pandoc@box{#1}%
  \Gscale@div\@tempa{\textheight}{\dimexpr\ht\pandoc@box+\dp\pandoc@box\relax}%
  \Gscale@div\@tempb{\linewidth}{\wd\pandoc@box}%
  \ifdim\@tempb\p@<\@tempa\p@\let\@tempa\@tempb\fi% select the smaller of both
  \ifdim\@tempa\p@<\p@\scalebox{\@tempa}{\usebox\pandoc@box}%
  \else\usebox{\pandoc@box}%
  \fi%
}
% Set default figure placement to htbp
\def\fps@figure{htbp}
\makeatother





\setlength{\emergencystretch}{3em} % prevent overfull lines

\providecommand{\tightlist}{%
  \setlength{\itemsep}{0pt}\setlength{\parskip}{0pt}}



 


%%%%%%%%%%%%%%%%%%%%
% start preamble.tex
%%%%%%%%%%%%%%%%%%%%

\input{resources/tex/boxes.tex}

% page layout
\geometry{
  dvips=false, pdftex=false, vtex=false, % drivers can have unexpected behaviors
  papersize={8in,10in},                  % size specified by MIT Press
  centering,                             % split margins equally
  margin=.6in,                           % margins (must all be at least .5in)
  includemp, includehead                % include sidenotes & header in body
  % showframe                              % show page structure (for debugging)
}

% set fonts
% \setmainfont[]{ETbb}
\setmainfont{ETbb}[
  UprightFont = {*-Regular},
  BoldFont = {*-Bold},
  ItalicFont = {*-Italic},
  BoldItalicFont = {*-BoldItalic},
  Path = {./resources/fonts/ETbb/},
  Extension = {.otf}
]

\setsansfont{SourceSansPro}[
  UprightFont = {*-Regular},
  % BoldFont = {*-Bold},
  % ItalicFont = {*-Italic},
  Path = {./resources/fonts/},
  Extension = {.ttf}
]

% set font specifications
\setkomafont{disposition}{\rmfamily\itshape}
\addtokomafont{part}{\sffamily\scshape}
\addtokomafont{partnumber}{\sffamily\scshape}
\addtokomafont{chapter}{\sffamily\scshape}
\setkomafont{partentry}{\sffamily\scshape}
\setkomafont{chapterentry}{\sffamily\scshape}
\addtokomafont{title}{\sffamily}%\scshape}
\addtokomafont{subtitle}{\sffamily}%\scshape}
% \addtokomafont{author}{\sffamily}
\addtokomafont{pagehead}{\sffamily\scshape}
\addtokomafont{pagenumber}{\sffamily\scshape}

\usepackage{amsmath}
\usepackage{unicode-math}

% adjust spacing around section headers
\RedeclareSectionCommand[
  runin=false,
  afterskip=0pt % remove extra space after for section
]{section}
\RedeclareSectionCommand[
  runin=false,
  afterskip=0pt % remove extra space after for subsection
]{subsection}

% only part number on part title pages
\renewcommand{\partformat}{\thepart}

% headers/footers
\usepackage{scrlayer-scrpage}
\KOMAoptions{headwidth=textwithmarginpar} % make header full width
\automark{chapter}
\clearpairofpagestyles
\renewcommand{\chaptermark}[1]{\markboth{#1}{}} % prevent chaptermark from uppercasing
\ihead{%
  \ifnum\value{chapter}>0 \thechapter\hspace{3pt} \fi % include chapter number if not 0
  \textsc{\leftmark} % then chapter name
}
\ohead{\pagemark}
\pagestyle{scrheadings}

% table of contents
\usepackage[titles]{tocloft}
\renewcommand{\cftpartfont}{\sffamily\scshape\Large}     % part title
\renewcommand{\cftpartpagefont}{\sffamily\scshape\large} % part page number
\setlength{\cftbeforepartskip}{1em}                   % part vspace before
\renewcommand{\cftchapfont}{\sffamily\scshape\large}     % chapter title
\renewcommand{\cftchappagefont}{\sffamily\scshape\large} % chapter page number
\setlength{\cftbeforechapskip}{.05em}                    % chapter vspace before

% set chapter numbers flushright
\newcommand{\chapnumlen}{.5em}
\renewcommand{\cftchappresnum}{\hfill}
\renewcommand{\cftchapaftersnum}{\hspace*{\chapnumlen}}
\addtolength{\cftchapnumwidth}{\chapnumlen}
% \renewcommand{\cftchapnumwidth}{\chapnumlen}
% \addtolength{\cftchapindent}{2em}

% \setlength{\cftbeforechapskip}{.25em}
% \setlength{\cftbeforepartskip}{1.5em}

\newcommand{\partnumlen}{.75em}
\renewcommand{\cftpartpresnum}{\hfill}
\renewcommand{\cftpartaftersnum}{\hspace*{\partnumlen}}
% \addtolength{\cftpartnumwidth}{\partnumlen}
\setlength{\cftpartindent}{0em}
% \renewcommand{\cftpartnumwidth}{\partnumlen}

% \renewcommand{\cftpartnumwidth}{\cftpartpagewidth}
% \renewcommand{\cftpartnumformat}[1]{\hfill{\bfseries #1}} % Adjust font weight/style if necessary
% \renewcommand{\cftpartnumwidth}{\numlen}  % Adjust this width as needed
% \renewcommand{\cftpartleader}{\hfill} % Use this to add the space before the number

% lists
\usepackage{enumitem}
\setlist[itemize]{
  label={--} % en-dash as bullet symbol
}

\usepackage{threeparttable} % for papaja apa tables
\setlength{\tabcolsep}{4pt} % horizontal space between table columns

% styling for captions
\usepackage[format=plain]{caption}
\usepackage{marginfix} % load before sidenotes to improve sidenote positioning
\usepackage{sidenotes}
\usepackage{marginnote}
\DeclareCaptionFont{caps}{\footnotesize}
\DeclareCaptionFont{labels}{\footnotesize\bfseries}

\captionsetup{
  labelfont=labels,
  textfont=caps,
  labelsep=newline,
  skip=0pt
}
\DeclareCaptionStyle{marginfigure}{labelfont=labels,textfont=caps,labelsep=newline,skip=3pt}
\DeclareCaptionStyle{sidecaption}{labelfont=labels,textfont=caps,labelsep=newline,aboveskip=-1em}

\setlength{\intextsep}{9pt}
\setlength{\textfloatsep}{6pt}

% reset sidenote counter at start of each chapter
\let\oldchapter\chapter
\def\chapter{%
  \setcounter{sidenote}{1}%
  \oldchapter
}


% space above and below equations
% \setlength{\abovedisplayskip}{0pt}
% \setlength{\belowdisplayskip}{0pt}
\usepackage[nodisplayskipstretch]{setspace}

 % override quarto box settings
\ifdefined\Shaded\renewenvironment{Shaded}
  {\begin{tcolorbox}[enhanced, borderline west={3pt}{0pt}{shadecolor}, breakable, interior hidden, frame hidden, boxrule=0pt, top=0pt, bottom=0pt, sharp corners]}
  {\end{tcolorbox}}
\fi

% index
\usepackage{imakeidx}
\makeindex[intoc=true] %, columns=3, columnseprule=true, options=-s latex/indexstyles.ist]

% try to prevent within-word hyphenation across pages
\brokenpenalty10000\relax

% temporary settings for copyediting
% \setstretch{2}
% \usepackage{lineno}
% \linenumbers

%%%%%%%%%%%%%%%%%%
% end preamble.tex
%%%%%%%%%%%%%%%%%%
\makeatletter
\@ifpackageloaded{tcolorbox}{}{\usepackage[skins,breakable]{tcolorbox}}
\@ifpackageloaded{fontawesome5}{}{\usepackage{fontawesome5}}
\definecolor{quarto-callout-color}{HTML}{909090}
\definecolor{quarto-callout-note-color}{HTML}{0758E5}
\definecolor{quarto-callout-important-color}{HTML}{CC1914}
\definecolor{quarto-callout-warning-color}{HTML}{EB9113}
\definecolor{quarto-callout-tip-color}{HTML}{00A047}
\definecolor{quarto-callout-caution-color}{HTML}{FC5300}
\definecolor{quarto-callout-color-frame}{HTML}{acacac}
\definecolor{quarto-callout-note-color-frame}{HTML}{4582ec}
\definecolor{quarto-callout-important-color-frame}{HTML}{d9534f}
\definecolor{quarto-callout-warning-color-frame}{HTML}{f0ad4e}
\definecolor{quarto-callout-tip-color-frame}{HTML}{02b875}
\definecolor{quarto-callout-caution-color-frame}{HTML}{fd7e14}
\makeatother
\makeatletter
\@ifpackageloaded{bookmark}{}{\usepackage{bookmark}}
\makeatother
\makeatletter
\@ifpackageloaded{caption}{}{\usepackage{caption}}
\AtBeginDocument{%
\ifdefined\contentsname
  \renewcommand*\contentsname{Table of contents}
\else
  \newcommand\contentsname{Table of contents}
\fi
\ifdefined\listfigurename
  \renewcommand*\listfigurename{List of Figures}
\else
  \newcommand\listfigurename{List of Figures}
\fi
\ifdefined\listtablename
  \renewcommand*\listtablename{List of Tables}
\else
  \newcommand\listtablename{List of Tables}
\fi
\ifdefined\figurename
  \renewcommand*\figurename{Figure}
\else
  \newcommand\figurename{Figure}
\fi
\ifdefined\tablename
  \renewcommand*\tablename{Table}
\else
  \newcommand\tablename{Table}
\fi
}
\@ifpackageloaded{float}{}{\usepackage{float}}
\floatstyle{ruled}
\@ifundefined{c@chapter}{\newfloat{codelisting}{h}{lop}}{\newfloat{codelisting}{h}{lop}[chapter]}
\floatname{codelisting}{Listing}
\newcommand*\listoflistings{\listof{codelisting}{List of Listings}}
\makeatother
\makeatletter
\makeatother
\makeatletter
\@ifpackageloaded{caption}{}{\usepackage{caption}}
\@ifpackageloaded{subcaption}{}{\usepackage{subcaption}}
\makeatother
\makeatletter
\@ifpackageloaded{sidenotes}{}{\usepackage{sidenotes}}
\@ifpackageloaded{marginnote}{}{\usepackage{marginnote}}
\makeatother
\usepackage{bookmark}
\IfFileExists{xurl.sty}{\usepackage{xurl}}{} % add URL line breaks if available
\urlstyle{same}
% Make links footnotes instead of hotlinks:
\DeclareRobustCommand{\href}[2]{#2\sidenote{\footnotesize \url{#1}}}
\hypersetup{
  pdftitle={Many Babies Manual},
  pdfauthor={Heidi Baumgartner},
  colorlinks=true,
  linkcolor={DarkBlue},
  filecolor={Maroon},
  citecolor={DarkGreen},
  urlcolor={DarkGreen},
  pdfcreator={LaTeX via pandoc}}


\title{Many Babies Manual}
\usepackage{etoolbox}
\makeatletter
\providecommand{\subtitle}[1]{% add subtitle to \maketitle
  \apptocmd{\@title}{\par {\large #1 \par}}{}{}
}
\makeatother
\subtitle{Policies, guidelines, and resources for the ManyBabies
Consortium}
\author{Heidi Baumgartner}
\date{2025-11-18}
\begin{document}
\newgeometry{}

\begin{titlepage}
\end{titlepage}

\begin{titlepage}
  \centering
  {\usekomafont{title}\scshape\Huge Many Babies Manual\par}\clearpage
\end{titlepage}

\begin{titlepage}
  \begin{center}
    {\usekomafont{title}\scshape\Huge Many Babies Manual\par}
    \vskip 1em
    {\usekomafont{subtitle}\LARGE Policies, guidelines, and resources
for the ManyBabies Consortium\par}
    \vskip 1em
    \setstretch{1.5}
    {\usekomafont{author}  and~Heidi Baumgartner \par}
    \vfill
    {\rmfamily\large The MIT Press\\Cambridge, Massachusetts\\London, England}
  \end{center}
\end{titlepage}

\begin{titlepage}
  \vspace*{\fill}
  {\rmfamily\scriptsize
    © 2025 Massachusetts Institute of Technology\par
    All rights reserved. No part of this book may be used to train artificial intelligence systems or reproduced in any form by any electronic or mechanical means (including photocopying, recording, or information storage and retrieval) without permission in writing from the publisher.\par
    The MIT Press would like to thank the anonymous peer reviewers who provided comments on drafts of this book. The generous work of academic experts is essential for establishing the authority and quality of our publications. We acknowledge with gratitude the contributions of these otherwise uncredited readers.\par
    This book was set in ETbb and Source Sans Pro by \_\_\_\_\_\_\_. Printed and bound in the United States of America.\par
    Library of Congress Cataloging-in-Publication Data is available.\par
    ISBN: 978-0-262-55256-1\par
    10 9 8 7 6 5 4 3 2 1\par
  }
  \vspace*{\fill}
\end{titlepage}

\restoregeometry{}
\renewcommand*\contentsname{Contents}
{
\hypersetup{linkcolor=}
\setcounter{tocdepth}{0}
\tableofcontents
}

\phantomsection\label{sec-preface}
\bookmarksetup{startatroot}

\chapter*{Preface}
\addcontentsline{toc}{chapter}{Preface}

\markboth{Preface}{Preface}

ManyBabies is a global consortium of developmental psychology
labs\ldots{}

\begin{tcolorbox}[colframe=.grey, title=\faPersonFallingBurst \enspace Accident report]

\subsection*{UNDER DEVELOPMENT}\label{under-development}
\addcontentsline{toc}{subsection}{UNDER DEVELOPMENT}

THIS MANUAL IS STILL UNDER DEVELOPMENT! Please use the Google Doc
version of the MB General Manual available from the Resources page of
the MB website for now. Thanks!

\end{tcolorbox}

\section*{How to use this manual}\label{how-to-use-this-manual}
\addcontentsline{toc}{section}{How to use this manual}

\markright{How to use this manual}

This manual is organized into

\section*{Onward!}\label{onward}
\addcontentsline{toc}{section}{Onward!}

\markright{Onward!}

Thanks for joining us at \emph{ManyBabies}! The full source code for the
manual is available at \url{https://github.com/manybabies/mb-manual}. We
encourage you to browse, comment, and log issues or
suggestions.\sidenote{\footnotesize The best way to give us specific feedback is to
  create an issue on our github page at
  \url{https://github.com/manybabies/mb-manual/issues}.}

\section*{Acknowledgments}\label{acknowledgments}
\addcontentsline{toc}{section}{Acknowledgments}

\markright{Acknowledgments}

This manual was adapted from \emph{Experimentology}, a open textbook of
experimental psychology methods {[}@experimentology2025{]}. We follow in
the footsteps of the \emph{Experimentology} authors by offering this
manual openly licensed via CC BY-NC 4.0.

Thanks to the contributions of countless ManyBabies members for their
efforts in developing the content of this manual.

\section*{References}

\begin{CSLReferences}{1}{0}



\end{CSLReferences}

\phantomsection\label{sec-codeofconduct}
\bookmarksetup{startatroot}

\chapter*{MB Code of Conduct}
\addcontentsline{toc}{chapter}{MB Code of Conduct}

\markboth{MB Code of Conduct}{MB Code of Conduct}

\begin{quote}
last updated: 07/07/2020
\end{quote}

\section*{Our Pledge}\label{our-pledge}
\addcontentsline{toc}{section}{Our Pledge}

\markright{Our Pledge}

In the interest of fostering an open and welcoming environment, we as
contributors and maintainers pledge to make participation in our project
and our community a harassment-free experience for everyone, regardless
of age, body size, disability, ethnicity, sex characteristics, gender
identity and expression, level of experience, education, socio-economic
status, nationality, political perspective, status within the community,
personal appearance, race, religion, or sexual identity and orientation.

\section*{Our Standards}\label{our-standards}
\addcontentsline{toc}{section}{Our Standards}

\markright{Our Standards}

Examples of behavior that contribute to creating a positive environment
include:

\begin{itemize}
\tightlist
\item
  Using welcoming and inclusive language
\item
  Being respectful of differing viewpoints and experiences
\item
  Gracefully accepting constructive criticism
\item
  Focusing on what is best for the community
\item
  Showing empathy towards other community members
\end{itemize}

Examples of unacceptable behavior by contributors include:

\begin{itemize}
\tightlist
\item
  The use of sexualized language or imagery and unwelcome sexual
  attention or advances
\item
  Trolling, insulting/derogatory comments, and personal or political
  attacks
\item
  Public or private harassment
\item
  Publishing others' private information, such as a physical or
  electronic address, without explicit permission
\item
  Other conduct which could reasonably be considered inappropriate in a
  professional setting
\end{itemize}

\section*{Our Responsibilities}\label{our-responsibilities}
\addcontentsline{toc}{section}{Our Responsibilities}

\markright{Our Responsibilities}

The Governing Board, Executive Director, and project leads are
responsible for clarifying the standards of acceptable behavior and are
expected to take appropriate and fair corrective action in response to
any instances of unacceptable behavior.

The Governing Board, Executive Director, and project leads have the
right and responsibility to remove, edit, or reject comments, commits,
code, wiki edits, issues, emails, and other contributions that are not
aligned to this Code of Conduct, or to ban temporarily or permanently
any contributor for other behaviors that they deem inappropriate,
threatening, offensive, or harmful.

\section*{Scope}\label{scope}
\addcontentsline{toc}{section}{Scope}

\markright{Scope}

This Code of Conduct applies both within project spaces (OSF, github,
mailing lists) and in public spaces when an individual is representing
the project or its community. Examples of representing a project or
community include using an official project e-mail address, posting via
an official social media account, or acting as an appointed
representative at an online or offline event (e.g.~during a conference
presentation or at a meetup). Representation of a project may be further
defined and clarified by the Governing Board.

\section*{Enforcement}\label{enforcement}
\addcontentsline{toc}{section}{Enforcement}

\markright{Enforcement}

Instances of abusive, harassing, or otherwise unacceptable behavior may
be reported by contacting the
\href{mailto:contact@manybabies.org}{ManyBabies Executive Director}, the
\href{mailto:govboard@manybabies.org}{ManyBabies Governing Board}, or
any of its \href{https://manybabies.org/people/}{members}. All
complaints will be reviewed and investigated and will result in a
response that is deemed necessary and appropriate to the circumstances,
ranging from discussion of these policies to expulsion from the group,
including all mailing lists (either temporarily or permanently). It is
at the complainant's discretion to pursue additional action through
other legal or institutional mechanisms. The Governing Board and
Executive Director are obligated to maintain confidentiality with regard
to the reporter of an incident. In cases of complaints directly to
project leads, Governing Board members, or the Executive Director, the
reporter's identity will be known only to those directly contacted by
the complainant, unless the complainant's explicit permission is
obtained.

Governing Board members or project leads who do not follow or enforce
the Code of Conduct in good faith may face temporary or permanent
repercussions as determined by other members of the Governing Board or
project's leadership.

Further details of specific enforcement policies may be posted
separately.

\section*{Attribution}\label{attribution}
\addcontentsline{toc}{section}{Attribution}

\markright{Attribution}

This Code of Conduct is adapted from the
\href{https://www.contributor-covenant.org/version/1/4/code-of-conduct.html}{Contributor
Covenant, version 1.4} and from the
\href{https://github.com/psych-ds/psych-DS/blob/master/CODE_OF_CONDUCT.md}{Psych-DS
Code of Conduct} (retrieved on 2018-11-23).

\section*{Contribute}\label{contribute}
\addcontentsline{toc}{section}{Contribute}

\markright{Contribute}

Note that this is a living document. Comments and questions are
explicitly encouraged.
\href{https://docs.google.com/document/d/1dZ3sF2UcxvpkfOfKSKFeObTMZRbpUYloMUiPYtZy0ng/edit?usp=sharing}{Here
is a version of the Code that allows comments}. \emph{(To make anonymous
comments, make sure you are not logged in to Google in your browser.)}

\part{New Projects}

\chapter{Proposing a project}\label{sec-proposal}

\chapter{Project Onboarding}\label{sec-projectonboarding}

\begin{quote}
The MB Executive Director will help project leads set up the following
resources before project launch:
\end{quote}

\begin{itemize}
\tightlist
\item
  \textbf{Name}: should be short and informative; use exisiting projects
  as a guide
\item
  \textbf{Logo}: main projects only; spin-offs can use main project logo
  or request a variant
\item
  \textbf{Directory on MB Google Drive}: project leads will be given
  full access and power to manage access/sharing
\item
  \textbf{Listserv}: \texttt{mb{[}proj\#{]}-list@manybabies.org}
\item
  \textbf{Contact email}: \texttt{mb{[}proj\#{]}@manybabies.org} (note
  that this address is set up as a Google Group with a
  \href{https://support.google.com/a/users/answer/10375787?hl=en}{collaborative
  inbox} for project leads to share)
\item
  \textbf{Page on MB website}: \texttt{manybabies.org/MB{[}proj\#{]}}
\end{itemize}

\chapter{Participation Call}\label{sec-participationcall}

\section{Project launch}\label{project-launch}

\begin{quote}
Project Leads will plan/coordinate project launch details in
consultation with the MB Exec. Dir. and Governing Board. These efforts
can include:
\end{quote}

\begin{itemize}
\tightlist
\item
  Announcement to MB community with invite to participate
\item
  Announcement shared to developmental listservs (ICIS, CDS)
\item
  Kick-off event
\end{itemize}

\bookmarksetup{startatroot}

\chapter{Conclusion}\label{sec-conclusion}

You've made it to the end of the ManyBabies Manual.

\section{Project-specific resources}\label{project-specific-resources}

\phantomsection\label{3ade8a4a-fb1d-4a6c-8409-ac45482d5fc9}



% \usepackage[left=1in,marginparwidth=2.0666666666667in,textwidth=4.1333333333333in,marginparsep=0.3in]{geometry}
% \index{independent variable|seealso{manipulation, treatment}}
% \index{manipulation|seealso{independent variable, treatment}}
% \index{treatment|seealso{independent variable, manipulation}}

% \index{dependent variable|seealso{measure, outcome}}
% \index{measure|seealso{dependent variable, outcome}}
% \index{outcome|seealso{dependent variable, measure}}

\index{de-identification|seealso{anonymization}}
\index{anonymization|seealso{de-identification}}

\index{analytic flexibility|seealso{p-hacking}}
\index{p-hacking|seealso{analytic flexibility}}

\index{publication bias|seealso{selective reporting}}
\index{selective reporting|seealso{publication bias}}

\index{Cohen's $d$|seealso{standardized mean difference (SMD)}}
\index{standardized mean difference (SMD)|seealso{Cohen's $d$}}

\index{multiplicity|seealso{multiple comparisons}}
\index{multiple comparisons|seealso{multiplicity}}

\index{robustness analysis|seealso{multiverse analysis, sensitivity analysis}}
\index{multiverse analysis|seealso{robustness analysis, sensitivity analysis}}
\index{sensitivity analysis|seealso{multiverse analysis, robustness analysis}}

% \index{APA|see{American Psychological Association (APA)}}
\index{CDI|see{Communicative Development Inventory}}
\index{DAG|see{directed acyclic graph (DAG)}}
\index{blinding|see{masking}}

\newgeometry{
  centering,                             % split margins equally
  margin=.6in,                           % margins (must all be at least .5in)
  includemp, includehead,                % include sidenotes & header in body
  % showframe                              % show page structure (for debugging)
  % left=1in,
  marginparwidth=0in,marginparsep=0.3in%,textwidth=4.1333333333333in
}

% \addtogeometry{}
\printindex
\restoregeometry{}


\end{document}
